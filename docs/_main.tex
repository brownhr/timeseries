% Options for packages loaded elsewhere
\PassOptionsToPackage{unicode}{hyperref}
\PassOptionsToPackage{hyphens}{url}
%
\documentclass[
]{book}
\usepackage{amsmath,amssymb}
\usepackage{lmodern}
\usepackage{iftex}
\ifPDFTeX
  \usepackage[T1]{fontenc}
  \usepackage[utf8]{inputenc}
  \usepackage{textcomp} % provide euro and other symbols
\else % if luatex or xetex
  \usepackage{unicode-math}
  \defaultfontfeatures{Scale=MatchLowercase}
  \defaultfontfeatures[\rmfamily]{Ligatures=TeX,Scale=1}
\fi
% Use upquote if available, for straight quotes in verbatim environments
\IfFileExists{upquote.sty}{\usepackage{upquote}}{}
\IfFileExists{microtype.sty}{% use microtype if available
  \usepackage[]{microtype}
  \UseMicrotypeSet[protrusion]{basicmath} % disable protrusion for tt fonts
}{}
\makeatletter
\@ifundefined{KOMAClassName}{% if non-KOMA class
  \IfFileExists{parskip.sty}{%
    \usepackage{parskip}
  }{% else
    \setlength{\parindent}{0pt}
    \setlength{\parskip}{6pt plus 2pt minus 1pt}}
}{% if KOMA class
  \KOMAoptions{parskip=half}}
\makeatother
\usepackage{xcolor}
\IfFileExists{xurl.sty}{\usepackage{xurl}}{} % add URL line breaks if available
\IfFileExists{bookmark.sty}{\usepackage{bookmark}}{\usepackage{hyperref}}
\hypersetup{
  pdftitle={Course Outline: Time Series Data in R},
  pdfauthor={Harrison Brown},
  hidelinks,
  pdfcreator={LaTeX via pandoc}}
\urlstyle{same} % disable monospaced font for URLs
\usepackage{color}
\usepackage{fancyvrb}
\newcommand{\VerbBar}{|}
\newcommand{\VERB}{\Verb[commandchars=\\\{\}]}
\DefineVerbatimEnvironment{Highlighting}{Verbatim}{commandchars=\\\{\}}
% Add ',fontsize=\small' for more characters per line
\usepackage{framed}
\definecolor{shadecolor}{RGB}{248,248,248}
\newenvironment{Shaded}{\begin{snugshade}}{\end{snugshade}}
\newcommand{\AlertTok}[1]{\textcolor[rgb]{0.94,0.16,0.16}{#1}}
\newcommand{\AnnotationTok}[1]{\textcolor[rgb]{0.56,0.35,0.01}{\textbf{\textit{#1}}}}
\newcommand{\AttributeTok}[1]{\textcolor[rgb]{0.77,0.63,0.00}{#1}}
\newcommand{\BaseNTok}[1]{\textcolor[rgb]{0.00,0.00,0.81}{#1}}
\newcommand{\BuiltInTok}[1]{#1}
\newcommand{\CharTok}[1]{\textcolor[rgb]{0.31,0.60,0.02}{#1}}
\newcommand{\CommentTok}[1]{\textcolor[rgb]{0.56,0.35,0.01}{\textit{#1}}}
\newcommand{\CommentVarTok}[1]{\textcolor[rgb]{0.56,0.35,0.01}{\textbf{\textit{#1}}}}
\newcommand{\ConstantTok}[1]{\textcolor[rgb]{0.00,0.00,0.00}{#1}}
\newcommand{\ControlFlowTok}[1]{\textcolor[rgb]{0.13,0.29,0.53}{\textbf{#1}}}
\newcommand{\DataTypeTok}[1]{\textcolor[rgb]{0.13,0.29,0.53}{#1}}
\newcommand{\DecValTok}[1]{\textcolor[rgb]{0.00,0.00,0.81}{#1}}
\newcommand{\DocumentationTok}[1]{\textcolor[rgb]{0.56,0.35,0.01}{\textbf{\textit{#1}}}}
\newcommand{\ErrorTok}[1]{\textcolor[rgb]{0.64,0.00,0.00}{\textbf{#1}}}
\newcommand{\ExtensionTok}[1]{#1}
\newcommand{\FloatTok}[1]{\textcolor[rgb]{0.00,0.00,0.81}{#1}}
\newcommand{\FunctionTok}[1]{\textcolor[rgb]{0.00,0.00,0.00}{#1}}
\newcommand{\ImportTok}[1]{#1}
\newcommand{\InformationTok}[1]{\textcolor[rgb]{0.56,0.35,0.01}{\textbf{\textit{#1}}}}
\newcommand{\KeywordTok}[1]{\textcolor[rgb]{0.13,0.29,0.53}{\textbf{#1}}}
\newcommand{\NormalTok}[1]{#1}
\newcommand{\OperatorTok}[1]{\textcolor[rgb]{0.81,0.36,0.00}{\textbf{#1}}}
\newcommand{\OtherTok}[1]{\textcolor[rgb]{0.56,0.35,0.01}{#1}}
\newcommand{\PreprocessorTok}[1]{\textcolor[rgb]{0.56,0.35,0.01}{\textit{#1}}}
\newcommand{\RegionMarkerTok}[1]{#1}
\newcommand{\SpecialCharTok}[1]{\textcolor[rgb]{0.00,0.00,0.00}{#1}}
\newcommand{\SpecialStringTok}[1]{\textcolor[rgb]{0.31,0.60,0.02}{#1}}
\newcommand{\StringTok}[1]{\textcolor[rgb]{0.31,0.60,0.02}{#1}}
\newcommand{\VariableTok}[1]{\textcolor[rgb]{0.00,0.00,0.00}{#1}}
\newcommand{\VerbatimStringTok}[1]{\textcolor[rgb]{0.31,0.60,0.02}{#1}}
\newcommand{\WarningTok}[1]{\textcolor[rgb]{0.56,0.35,0.01}{\textbf{\textit{#1}}}}
\usepackage{longtable,booktabs,array}
\usepackage{calc} % for calculating minipage widths
% Correct order of tables after \paragraph or \subparagraph
\usepackage{etoolbox}
\makeatletter
\patchcmd\longtable{\par}{\if@noskipsec\mbox{}\fi\par}{}{}
\makeatother
% Allow footnotes in longtable head/foot
\IfFileExists{footnotehyper.sty}{\usepackage{footnotehyper}}{\usepackage{footnote}}
\makesavenoteenv{longtable}
\usepackage{graphicx}
\makeatletter
\def\maxwidth{\ifdim\Gin@nat@width>\linewidth\linewidth\else\Gin@nat@width\fi}
\def\maxheight{\ifdim\Gin@nat@height>\textheight\textheight\else\Gin@nat@height\fi}
\makeatother
% Scale images if necessary, so that they will not overflow the page
% margins by default, and it is still possible to overwrite the defaults
% using explicit options in \includegraphics[width, height, ...]{}
\setkeys{Gin}{width=\maxwidth,height=\maxheight,keepaspectratio}
% Set default figure placement to htbp
\makeatletter
\def\fps@figure{htbp}
\makeatother
\setlength{\emergencystretch}{3em} % prevent overfull lines
\providecommand{\tightlist}{%
  \setlength{\itemsep}{0pt}\setlength{\parskip}{0pt}}
\setcounter{secnumdepth}{5}
\usepackage{booktabs}
\ifLuaTeX
  \usepackage{selnolig}  % disable illegal ligatures
\fi
\usepackage[]{natbib}
\bibliographystyle{plainnat}

\title{Course Outline: Time Series Data in R}
\author{Harrison Brown}
\date{2022-05-08}

\begin{document}
\maketitle

{
\setcounter{tocdepth}{1}
\tableofcontents
}
\hypertarget{welcome}{%
\chapter*{Welcome}\label{welcome}}
\addcontentsline{toc}{chapter}{Welcome}

Welcome to the course outline for \emph{Time Series Data in R}! This course offers methods and workflows for analyzing and interpreting time series data, an overview of when, why, and how to use time series data, and various utilities and packages in R that are beneficial to analysts.

By the end of this course, students will have the skills to:

\begin{itemize}
\tightlist
\item
  Interpret and understand time series plots
\item
  Import ts data to create and manipulate \texttt{ts} objects from the \texttt{stats} package
\item
  Understand why time series data is fundamentally different than non-ts data.
\item
  Analyze time series data with plots
\item
  ?Intro to Wavelet analysis?
\end{itemize}

\hypertarget{introduction-to-time-series-data}{%
\chapter{Introduction to time series data}\label{introduction-to-time-series-data}}

\hypertarget{lesson-what-is-time-series-data}{%
\section{Lesson: What is Time Series Data}\label{lesson-what-is-time-series-data}}

\begin{itemize}
\tightlist
\item
  Learning Objective: Learner will be able to understand why and how TS-data differs from non-temporal data
\item
  LO: What kinds of inferences and results can be obtained from TS-data
\item
  LO: Converting to and from time-based data formats, such as \texttt{numeric}, \texttt{Date}, and \texttt{POSIXct} classes

  \begin{itemize}
  \tightlist
  \item
    Functions: \texttt{as.Date()}, \texttt{lubridate::}, etc.
  \end{itemize}
\end{itemize}

\hypertarget{lesson-how-to-interpret-time-series-data}{%
\section{Lesson: How to Interpret Time Series Data}\label{lesson-how-to-interpret-time-series-data}}

\begin{itemize}
\tightlist
\item
  LO: Learner will understand how to interpret attributes of a basic time series plot
\item
  LO: ``Signal and Noise'' in the context of TS data
\item
  Introduction to Stationarity: Most real-world data are not stationary and require additional steps to work with
\end{itemize}

\hypertarget{lesson-components-of-time-series-data}{%
\section{Lesson: Components of Time Series Data}\label{lesson-components-of-time-series-data}}

\hypertarget{creating-and-manipulating-time-series}{%
\chapter{Creating and Manipulating Time Series}\label{creating-and-manipulating-time-series}}

\hypertarget{ts-class}{%
\section{\texorpdfstring{\texttt{ts} Class}{ts Class}}\label{ts-class}}

\hypertarget{missing-values}{%
\section{Missing Values}\label{missing-values}}

\begin{Shaded}
\begin{Highlighting}[]
\NormalTok{d }\OtherTok{\textless{}{-}} \FunctionTok{c}\NormalTok{(}\StringTok{\textquotesingle{}2001{-}01{-}01\textquotesingle{}}\NormalTok{, }\StringTok{\textquotesingle{}2001{-}01{-}02\textquotesingle{}}\NormalTok{, }\StringTok{\textquotesingle{}2001{-}01{-}04\textquotesingle{}}\NormalTok{, }\StringTok{\textquotesingle{}2001{-}01{-}05\textquotesingle{}}\NormalTok{)}
\NormalTok{d }\OtherTok{\textless{}{-}} \FunctionTok{as.Date}\NormalTok{(d)}
\NormalTok{date\_range }\OtherTok{\textless{}{-}} \FunctionTok{seq}\NormalTok{(}\FunctionTok{min}\NormalTok{(d), }\FunctionTok{max}\NormalTok{(d), }\AttributeTok{by =} \DecValTok{1}\NormalTok{) }
\NormalTok{date\_range[}\SpecialCharTok{!}\NormalTok{date\_range }\SpecialCharTok{\%in\%}\NormalTok{ d] }
\end{Highlighting}
\end{Shaded}

\begin{verbatim}
## [1] "2001-01-03"
\end{verbatim}

\hypertarget{rolling-and-expanding-windows}{%
\chapter{Rolling and Expanding Windows}\label{rolling-and-expanding-windows}}

\hypertarget{rolling-window}{%
\section{Rolling Window}\label{rolling-window}}

\begin{itemize}
\tightlist
\item
  Moving lower and upper bound
\end{itemize}

\hypertarget{data}{%
\subsection{Data}\label{data}}

\hypertarget{calculating-a-rolling-window}{%
\subsection{Calculating a Rolling Window}\label{calculating-a-rolling-window}}

\hypertarget{introduction-to-forecasting-in-r}{%
\chapter{Introduction to Forecasting in R}\label{introduction-to-forecasting-in-r}}

\hypertarget{methods-for-forecasting}{%
\section{Methods for Forecasting}\label{methods-for-forecasting}}

\hypertarget{exponential-smoothing}{%
\subsection{Exponential Smoothing}\label{exponential-smoothing}}

\hypertarget{capstone-exercise}{%
\chapter{Capstone Exercise}\label{capstone-exercise}}

The final exercise for this course involves performing a time series analysis on real-world data: Carbon Dioxide concentration at the Mauna Loa Observatory, from early 1959 to Present. You'll go through the process of imputing missing values, testing for stationarity, decomposing the time series, and adjusting for seasonality. The goal for this exercise is a plot showing the seasonal-adjusted time series, which shows the ``overall trend'' in the data over the last few decades.

\hypertarget{importing-the-data}{%
\section{Importing the Data}\label{importing-the-data}}

\begin{Shaded}
\begin{Highlighting}[]
\CommentTok{\# The following libraries are included for you}

\FunctionTok{library}\NormalTok{(tidyverse)}
\FunctionTok{library}\NormalTok{(zoo)}
\FunctionTok{library}\NormalTok{(forecast)}
\FunctionTok{library}\NormalTok{(tseries)}
\CommentTok{\# Sample data from the Mauna Loa Observatory}
\CommentTok{\# https://gml.noaa.gov/webdata/ccgg/trends/co2/co2\_mm\_mlo.csv}

\CommentTok{\# Data is already pre{-}processed as a \textasciigrave{}ts\textasciigrave{} object. It contains missing values, so}
\CommentTok{\# we\textquotesingle{}ll need to impute those!}
\NormalTok{co2 }\OtherTok{\textless{}{-}} \FunctionTok{readRDS}\NormalTok{(}\StringTok{"data/missing.Rds"}\NormalTok{)}
\end{Highlighting}
\end{Shaded}

\hypertarget{visual-checks}{%
\section{Visual Checks}\label{visual-checks}}

\begin{enumerate}
\def\labelenumi{\arabic{enumi}.}
\tightlist
\item
  Plot your co2 data and see if the data is stationary, non-stationary (time-dependent), and seasonal or non-seasonal. Use the Augmented Dickey-Fuller test to determine stationarity.
\end{enumerate}

\begin{Shaded}
\begin{Highlighting}[]
\FunctionTok{plot.ts}\NormalTok{(co2)}
\end{Highlighting}
\end{Shaded}

\begin{center}\includegraphics{_main_files/figure-latex/visual-1} \end{center}

\begin{Shaded}
\begin{Highlighting}[]
\FunctionTok{adf.test}\NormalTok{(co2)}
\end{Highlighting}
\end{Shaded}

\begin{verbatim}
## Error in adf.test(co2): NAs in x
\end{verbatim}

Looks like we have some missing values! Let's try and impute those to fill in the gaps:

\begin{enumerate}
\def\labelenumi{\arabic{enumi}.}
\setcounter{enumi}{1}
\tightlist
\item
  Impute missing values with the \texttt{na.approx()} function from \texttt{zoo}:
\end{enumerate}

\begin{Shaded}
\begin{Highlighting}[]
\NormalTok{co2 }\OtherTok{\textless{}{-}}\NormalTok{ co2 }\SpecialCharTok{\%\textgreater{}\%} 
\NormalTok{  zoo}\SpecialCharTok{::}\FunctionTok{na.approx}\NormalTok{()}

\FunctionTok{adf.test}\NormalTok{(co2)}
\end{Highlighting}
\end{Shaded}

\begin{verbatim}
## Warning in adf.test(co2): p-value greater than printed p-value
\end{verbatim}

\begin{verbatim}
## 
##  Augmented Dickey-Fuller Test
## 
## data:  co2
## Dickey-Fuller = -0.12404, Lag order = 9, p-value = 0.99
## alternative hypothesis: stationary
\end{verbatim}

Based on the statistic and p-value, the data is non-stationary.

\hypertarget{decomposing-the-time-series}{%
\section{Decomposing the Time Series}\label{decomposing-the-time-series}}

We're interested in the seasonal, remainder, and specifically, trend components. We want to know what the data looks like when adjusted for seasonality. To do this, we need to decompose our time series into its ETS components. Then, we can remove the seasonal component.

\begin{enumerate}
\def\labelenumi{\arabic{enumi}.}
\setcounter{enumi}{2}
\tightlist
\item
  Decompose the time series and plot the resulting \texttt{decomposed.ts} object:
\end{enumerate}

\begin{Shaded}
\begin{Highlighting}[]
\NormalTok{co2\_decomp }\OtherTok{\textless{}{-}}\NormalTok{ co2 }\SpecialCharTok{\%\textgreater{}\%}
  \FunctionTok{decompose}\NormalTok{()}

\FunctionTok{plot}\NormalTok{(co2\_decomp)}
\end{Highlighting}
\end{Shaded}

\begin{center}\includegraphics{_main_files/figure-latex/decomp-1} \end{center}

\begin{enumerate}
\def\labelenumi{\arabic{enumi}.}
\setcounter{enumi}{3}
\tightlist
\item
  Then, adjust for seasonality, and plot the results. Label the y-axis as ``CO2 Concentration'', and title the plot ``Seasonally-adjusted CO2 Concentration'':
\end{enumerate}

\begin{Shaded}
\begin{Highlighting}[]
\NormalTok{co2\_decomp }\SpecialCharTok{\%\textgreater{}\%} 
  \FunctionTok{seasadj}\NormalTok{() }\SpecialCharTok{\%\textgreater{}\%} 
  
  \CommentTok{\# If we want to include a rolling window, it\textquotesingle{}s really easy to do with}
  \CommentTok{\# zoo::rollapplyr()}
  
  \CommentTok{\#rollapplyr(FUN = mean, width = 12) \%\textgreater{}\%}
  \FunctionTok{plot}\NormalTok{(}\AttributeTok{ylab =} \StringTok{"CO2 Concentration"}\NormalTok{, }\AttributeTok{main =} \StringTok{"Seasonally{-}adjusted CO2 Concentration"}\NormalTok{)}
\end{Highlighting}
\end{Shaded}

\begin{center}\includegraphics{_main_files/figure-latex/unnamed-chunk-7-1} \end{center}

Voila! We now have the overall trend, plus remainder component, of CO2 concentrations. By removing the seasonality of data, we can better assess the year-to-year differences in our data, and can reduce the effect of intra- and inter-year cycles present in the data.

\hypertarget{example-code}{%
\section{Example Code}\label{example-code}}

This is the code given in the exercise itself

Step 1:

\begin{Shaded}
\begin{Highlighting}[]
\FunctionTok{plot}\NormalTok{(\_\_\_)}

\FunctionTok{\_\_\_.test}\NormalTok{(co2)}
\end{Highlighting}
\end{Shaded}

Step 2:

\begin{Shaded}
\begin{Highlighting}[]
\NormalTok{co2\_impute }\OtherTok{\textless{}{-}}\NormalTok{ co2 }\SpecialCharTok{\%\textgreater{}\%} 
  \FunctionTok{\_\_.\_\_\_}\NormalTok{()}

\FunctionTok{adf.test}\NormalTok{(co2\_impute)}
\end{Highlighting}
\end{Shaded}

Step 3:

\begin{Shaded}
\begin{Highlighting}[]
\NormalTok{co2\_decomp }\OtherTok{\textless{}{-}}\NormalTok{ co2\_impute }\SpecialCharTok{\%\textgreater{}\%} 
  \FunctionTok{\_\_\_}\NormalTok{()}

\FunctionTok{plot}\NormalTok{(\_\_\_)}
\end{Highlighting}
\end{Shaded}

Step 4:

\begin{Shaded}
\begin{Highlighting}[]
\NormalTok{co2\_decomp }\SpecialCharTok{\%\textgreater{}\%} 
  \FunctionTok{\_\_\_\_}\NormalTok{() }\SpecialCharTok{\%\textgreater{}\%} 
  \FunctionTok{plot}\NormalTok{(}\AttributeTok{ylab =} \StringTok{"\_\_\_"}\NormalTok{, }\AttributeTok{main =} \StringTok{"\_\_\_"}\NormalTok{)}
\end{Highlighting}
\end{Shaded}


  \bibliography{book.bib,packages.bib}

\end{document}
