% Options for packages loaded elsewhere
\PassOptionsToPackage{unicode}{hyperref}
\PassOptionsToPackage{hyphens}{url}
%
\documentclass[
]{book}
\usepackage{amsmath,amssymb}
\usepackage{lmodern}
\usepackage{iftex}
\ifPDFTeX
  \usepackage[T1]{fontenc}
  \usepackage[utf8]{inputenc}
  \usepackage{textcomp} % provide euro and other symbols
\else % if luatex or xetex
  \usepackage{unicode-math}
  \defaultfontfeatures{Scale=MatchLowercase}
  \defaultfontfeatures[\rmfamily]{Ligatures=TeX,Scale=1}
\fi
% Use upquote if available, for straight quotes in verbatim environments
\IfFileExists{upquote.sty}{\usepackage{upquote}}{}
\IfFileExists{microtype.sty}{% use microtype if available
  \usepackage[]{microtype}
  \UseMicrotypeSet[protrusion]{basicmath} % disable protrusion for tt fonts
}{}
\makeatletter
\@ifundefined{KOMAClassName}{% if non-KOMA class
  \IfFileExists{parskip.sty}{%
    \usepackage{parskip}
  }{% else
    \setlength{\parindent}{0pt}
    \setlength{\parskip}{6pt plus 2pt minus 1pt}}
}{% if KOMA class
  \KOMAoptions{parskip=half}}
\makeatother
\usepackage{xcolor}
\IfFileExists{xurl.sty}{\usepackage{xurl}}{} % add URL line breaks if available
\IfFileExists{bookmark.sty}{\usepackage{bookmark}}{\usepackage{hyperref}}
\hypersetup{
  pdftitle={Course Outline: Time Series Data in R},
  pdfauthor={Harrison Brown},
  hidelinks,
  pdfcreator={LaTeX via pandoc}}
\urlstyle{same} % disable monospaced font for URLs
\usepackage{color}
\usepackage{fancyvrb}
\newcommand{\VerbBar}{|}
\newcommand{\VERB}{\Verb[commandchars=\\\{\}]}
\DefineVerbatimEnvironment{Highlighting}{Verbatim}{commandchars=\\\{\}}
% Add ',fontsize=\small' for more characters per line
\usepackage{framed}
\definecolor{shadecolor}{RGB}{248,248,248}
\newenvironment{Shaded}{\begin{snugshade}}{\end{snugshade}}
\newcommand{\AlertTok}[1]{\textcolor[rgb]{0.94,0.16,0.16}{#1}}
\newcommand{\AnnotationTok}[1]{\textcolor[rgb]{0.56,0.35,0.01}{\textbf{\textit{#1}}}}
\newcommand{\AttributeTok}[1]{\textcolor[rgb]{0.77,0.63,0.00}{#1}}
\newcommand{\BaseNTok}[1]{\textcolor[rgb]{0.00,0.00,0.81}{#1}}
\newcommand{\BuiltInTok}[1]{#1}
\newcommand{\CharTok}[1]{\textcolor[rgb]{0.31,0.60,0.02}{#1}}
\newcommand{\CommentTok}[1]{\textcolor[rgb]{0.56,0.35,0.01}{\textit{#1}}}
\newcommand{\CommentVarTok}[1]{\textcolor[rgb]{0.56,0.35,0.01}{\textbf{\textit{#1}}}}
\newcommand{\ConstantTok}[1]{\textcolor[rgb]{0.00,0.00,0.00}{#1}}
\newcommand{\ControlFlowTok}[1]{\textcolor[rgb]{0.13,0.29,0.53}{\textbf{#1}}}
\newcommand{\DataTypeTok}[1]{\textcolor[rgb]{0.13,0.29,0.53}{#1}}
\newcommand{\DecValTok}[1]{\textcolor[rgb]{0.00,0.00,0.81}{#1}}
\newcommand{\DocumentationTok}[1]{\textcolor[rgb]{0.56,0.35,0.01}{\textbf{\textit{#1}}}}
\newcommand{\ErrorTok}[1]{\textcolor[rgb]{0.64,0.00,0.00}{\textbf{#1}}}
\newcommand{\ExtensionTok}[1]{#1}
\newcommand{\FloatTok}[1]{\textcolor[rgb]{0.00,0.00,0.81}{#1}}
\newcommand{\FunctionTok}[1]{\textcolor[rgb]{0.00,0.00,0.00}{#1}}
\newcommand{\ImportTok}[1]{#1}
\newcommand{\InformationTok}[1]{\textcolor[rgb]{0.56,0.35,0.01}{\textbf{\textit{#1}}}}
\newcommand{\KeywordTok}[1]{\textcolor[rgb]{0.13,0.29,0.53}{\textbf{#1}}}
\newcommand{\NormalTok}[1]{#1}
\newcommand{\OperatorTok}[1]{\textcolor[rgb]{0.81,0.36,0.00}{\textbf{#1}}}
\newcommand{\OtherTok}[1]{\textcolor[rgb]{0.56,0.35,0.01}{#1}}
\newcommand{\PreprocessorTok}[1]{\textcolor[rgb]{0.56,0.35,0.01}{\textit{#1}}}
\newcommand{\RegionMarkerTok}[1]{#1}
\newcommand{\SpecialCharTok}[1]{\textcolor[rgb]{0.00,0.00,0.00}{#1}}
\newcommand{\SpecialStringTok}[1]{\textcolor[rgb]{0.31,0.60,0.02}{#1}}
\newcommand{\StringTok}[1]{\textcolor[rgb]{0.31,0.60,0.02}{#1}}
\newcommand{\VariableTok}[1]{\textcolor[rgb]{0.00,0.00,0.00}{#1}}
\newcommand{\VerbatimStringTok}[1]{\textcolor[rgb]{0.31,0.60,0.02}{#1}}
\newcommand{\WarningTok}[1]{\textcolor[rgb]{0.56,0.35,0.01}{\textbf{\textit{#1}}}}
\usepackage{longtable,booktabs,array}
\usepackage{calc} % for calculating minipage widths
% Correct order of tables after \paragraph or \subparagraph
\usepackage{etoolbox}
\makeatletter
\patchcmd\longtable{\par}{\if@noskipsec\mbox{}\fi\par}{}{}
\makeatother
% Allow footnotes in longtable head/foot
\IfFileExists{footnotehyper.sty}{\usepackage{footnotehyper}}{\usepackage{footnote}}
\makesavenoteenv{longtable}
\usepackage{graphicx}
\makeatletter
\def\maxwidth{\ifdim\Gin@nat@width>\linewidth\linewidth\else\Gin@nat@width\fi}
\def\maxheight{\ifdim\Gin@nat@height>\textheight\textheight\else\Gin@nat@height\fi}
\makeatother
% Scale images if necessary, so that they will not overflow the page
% margins by default, and it is still possible to overwrite the defaults
% using explicit options in \includegraphics[width, height, ...]{}
\setkeys{Gin}{width=\maxwidth,height=\maxheight,keepaspectratio}
% Set default figure placement to htbp
\makeatletter
\def\fps@figure{htbp}
\makeatother
\setlength{\emergencystretch}{3em} % prevent overfull lines
\providecommand{\tightlist}{%
  \setlength{\itemsep}{0pt}\setlength{\parskip}{0pt}}
\setcounter{secnumdepth}{5}
\usepackage{booktabs}
\ifLuaTeX
  \usepackage{selnolig}  % disable illegal ligatures
\fi
\usepackage[]{natbib}
\bibliographystyle{plainnat}

\title{Course Outline: Time Series Data in R}
\author{Harrison Brown}
\date{2022-05-06}

\begin{document}
\maketitle

{
\setcounter{tocdepth}{1}
\tableofcontents
}
\hypertarget{welcome}{%
\chapter*{Welcome}\label{welcome}}
\addcontentsline{toc}{chapter}{Welcome}

Welcome to the course outline for \emph{Time Series Data in R}! This course offers methods and workflows for analyzing and interpreting time series data, an overview of when, why, and how to use time series data, and various utilities and packages in R that are beneficial to analysts.

By the end of this course, students will have the skills to:

\begin{itemize}
\tightlist
\item
  Interpret and understand time series plots
\item
  Import ts data to create and manipulate \texttt{ts} objects from the \texttt{stats} package
\item
  Understand why time series data is fundamentally different than non-ts data.
\item
  Analyze time series data with plots
\item
  ?Intro to Wavelet analysis?
\end{itemize}

\hypertarget{introduction-to-time-series-data}{%
\chapter{Introduction to time series data}\label{introduction-to-time-series-data}}

\hypertarget{lesson-what-is-time-series-data}{%
\section{Lesson: What is Time Series Data}\label{lesson-what-is-time-series-data}}

\begin{itemize}
\tightlist
\item
  Learning Objective: Learner will be able to understand why and how TS-data differs from non-temporal data
\item
  LO: What kinds of inferences and results can be obtained from TS-data
\item
  LO: Converting to and from time-based data formats, such as \texttt{numeric}, \texttt{Date}, and \texttt{POSIXct} classes

  \begin{itemize}
  \tightlist
  \item
    Functions: \texttt{as.Date()}, \texttt{lubridate::}, etc.
  \end{itemize}
\end{itemize}

\hypertarget{lesson-how-to-interpret-time-series-data}{%
\section{Lesson: How to Interpret Time Series Data}\label{lesson-how-to-interpret-time-series-data}}

\begin{itemize}
\tightlist
\item
  LO: Learner will understand how to interpret attributes of a basic time series plot
\item
  LO: ``Signal and Noise'' in the context of TS data
\item
  Introduction to Stationarity: Most real-world data are not stationary and require additional steps to work with
\end{itemize}

\hypertarget{lesson-components-of-time-series-data}{%
\section{Lesson: Components of Time Series Data}\label{lesson-components-of-time-series-data}}

\hypertarget{creating-and-manipulating-time-series}{%
\chapter{Creating and Manipulating Time Series}\label{creating-and-manipulating-time-series}}

\hypertarget{ts-class}{%
\section{\texorpdfstring{\texttt{ts} Class}{ts Class}}\label{ts-class}}

\hypertarget{creating-a-ts.plot}{%
\section{\texorpdfstring{Creating a \texttt{ts.plot()}}{Creating a ts.plot()}}\label{creating-a-ts.plot}}

\hypertarget{interpreting-plots}{%
\subsection{Interpreting Plots}\label{interpreting-plots}}

\hypertarget{seasonality-plot}{%
\subsection{Seasonality Plot}\label{seasonality-plot}}

\begin{Shaded}
\begin{Highlighting}[]
\FunctionTok{ggseasonplot}\NormalTok{(}\AttributeTok{x =}\NormalTok{ AirPassengers)}
\end{Highlighting}
\end{Shaded}

\begin{center}\includegraphics{_main_files/figure-latex/season-1-1} \end{center}

\hypertarget{polar-seasonality-plot}{%
\subsection{Polar Seasonality Plot}\label{polar-seasonality-plot}}

\hypertarget{trends-and-seasons}{%
\section{Trends and Seasons}\label{trends-and-seasons}}

\hypertarget{decomposition}{%
\subsection{Decomposition}\label{decomposition}}

\hypertarget{de-trending-data}{%
\subsection{De-trending Data}\label{de-trending-data}}

\hypertarget{rolling-and-expanding-windows}{%
\chapter{Rolling and Expanding Windows}\label{rolling-and-expanding-windows}}

\hypertarget{rolling-window}{%
\section{Rolling Window}\label{rolling-window}}

\begin{itemize}
\tightlist
\item
  Moving lower and upper bound
\end{itemize}

\hypertarget{data}{%
\subsection{Data}\label{data}}

\hypertarget{calculating-a-rolling-window}{%
\subsection{Calculating a Rolling Window}\label{calculating-a-rolling-window}}

\hypertarget{introduction-to-forecasting-in-r}{%
\chapter{Introduction to Forecasting in R}\label{introduction-to-forecasting-in-r}}

\hypertarget{methods-for-forecasting}{%
\section{Methods for Forecasting}\label{methods-for-forecasting}}

\hypertarget{exponential-smoothing}{%
\subsection{Exponential Smoothing}\label{exponential-smoothing}}

\hypertarget{final-exercise}{%
\chapter{Final Exercise}\label{final-exercise}}

The final exercise for this course involves performing a time series analysis on real-world sales data. You'll go step-by-step from reading the data and checking attributes like stationarity, to normalizing, decomposing, adjusting, and interpreting the results.

\hypertarget{importing-the-data}{%
\section{Importing the Data}\label{importing-the-data}}

\hypertarget{visual-checks}{%
\section{Visual Checks}\label{visual-checks}}

\begin{Shaded}
\begin{Highlighting}[]
\NormalTok{sales\_ts }\OtherTok{\textless{}{-}}\NormalTok{ tsbox}\SpecialCharTok{::}\FunctionTok{ts\_ts}\NormalTok{(sales)}
\end{Highlighting}
\end{Shaded}

\begin{verbatim}
## [time]: 'date' [value]: 'sales'
\end{verbatim}

\begin{Shaded}
\begin{Highlighting}[]
\FunctionTok{autoplot.zoo}\NormalTok{(sales\_ts)}
\end{Highlighting}
\end{Shaded}

\begin{center}\includegraphics{_main_files/figure-latex/unnamed-chunk-4-1} \end{center}

\hypertarget{quarterly-summary}{%
\subsection{Quarterly summary}\label{quarterly-summary}}

  \bibliography{book.bib,packages.bib}

\end{document}
